%!TEX root=../Vorlage_DA.tex
%	########################################################
% 					Richtlinien für Diplomarbeit
%	########################################################


%	--------------------------------------------------------
% 	Überschrift, Inhaltsverzeichnis
%	--------------------------------------------------------
\chapter{Richtlinien für die Erstellung von Diplomarbeiten und Projektdokumentationen}
\chaptermark{Richtlinien für die Erstellung}


%	--------------------------------------------------------
% 	Allgmeine Hinweise
%	--------------------------------------------------------
\section{Allgemeine Hinweise}

Die Ausgabe des Themas der Diplomarbeit erfolgt spätestens 8 Wochen nach Schulbeginn. Die Arbeit ist gebunden (z.B. Spiralbindung) in dreifacher Ausfertigung abzugeben. 

Jedes Exemplar muss eine unterschriebene Versicherung enthalten, dass die Arbeit selb\-ständig verfasst und keine anderen als die angegebenen Quellen und Hilfsmittel benutzt wurden. Bei einer Gruppenarbeit ist die Angabe des jeweiligen Beitrags des Einzelnen erforderlich.

Bei Verwendung dieser Vorlage sind die nicht zutreffenden Bezeichnungen, Teile u.ä. wegzulassen (z.B. nur Projektarbeit oder Diplomarbeit im Titel), Hinweise durch entsprechende Inhalte zu ersetzen (z.B. Fach / Fächer) und Angaben nach Bedarf mehrfach zu geben (z.B. über die Autoren bei mehreren Autoren).

Diese und die folgenden Hinweise gelten auch für Abschlussarbeiten (unter Berücksichtigung der durch die Projektbetreuer definierten Einschränkungen).

%	--------------------------------------------------------
% 	Inhaltliche Gestaltung
%	--------------------------------------------------------
\section{Inhaltliche Gestaltung}

Die Diplomarbeit soll zeigen, dass der Kandidat in der Lage ist, innerhalb einer vorgegebenen Frist eine fächerübergreifende Aufgabe selb\-ständig zu bearbeiten. Die Formulierung soll sachlich und präzise sein. Verwenden Sie klare Hauptsätze mit einfachen Nebensätzen und keine Schachtelsätze. Die \emph{Ich-} und \emph{Wir-Form} ist zu vermeiden, ebenso Allgemeinplätze wie z.B. \emph{\glqq Wie allgemein bekannt...\grqq}. 

\subsection{Umfang}

Ein in Seiten gemessener Mindest- oder Höchstumfang der Arbeit ist nicht möglich. Der übliche Umfang beträgt etwa 40 - 60 Seiten. Die Ausführlichkeit des Textes sollte so bemessen sein, dass der Problemkreis von einem fachkundigen, auf das betreffende Teilgebiet jedoch nicht spezialisierten Leser verstanden werden kann. 

\emph{\glqq Qualität geht vor Quantität.\grqq}


\subsection{Vorwort}

Im Vorwort teilt der Bearbeiter dem Leser wichtige Tatsachen mit, die Erklärungen zu seiner Arbeit beinhalten; z. B. die Motivation für die Bearbeitung des Themas oder besondere Schwierigkeiten bei der Bearbeitung und/oder Materialbeschaffung. Hier können auch Mitteilungen persönlicher Natur enthalten sein; z. B. Dank an Institutionen/Personen für die geleistete Unterstützung. 

\subsection{Anhang}

Im Anhang sind das Projekttagebuch und gegebenenfalls Schaltpläne, Quellcodes, Messprotokolle, Datenblätter u.ä. anzuführen. 

%	--------------------------------------------------------
% 	Zitierweise
%	--------------------------------------------------------
\section{Zitierweise}

Zitate aus der Sekundärliteratur sind möglichst zu vermeiden. Sind sie unumgänglich, sind sie durch den Hinweis \glqq zit.\grqq mit Angabe der Sekundärquelle kenntlich zu machen. 

Es ist zunehmend eine Kurzzitierweise in Fußnoten üblich: 
Nachname des Verfassers, Kurztitel, Seitenangabe. 

\subsection{Wörtliche Zitate}
Wörtliche Zitate werden durch Anführungszeichen begonnen und beendet und kursiv geschrieben. Sie sind sparsam zu verwenden. Will man einen Satz nicht vollständig wiedergeben, hat man die Auslassung durch Punkte (\ldots) anzuzeigen. Zitate im Zitat werden durch Apostroph  am Anfang (\glq) und  Ende (\grq) kenntlich gemacht. 

\subsection{Sinngemäße Zitate und Anlehnungen}
Das sinngemäße Zitat hat den Zweck, die Gedanken, nicht die Worte, eines Autors wiederzugeben. Die Formulierungen sind so zu wählen, dass für jeden Teil der Aussage erkenntlich ist, wessen Meinung vorgetragen wird. 

\subsection{Abbildungen}

Abbildungen sind laufend zu nummerieren und mit einer Bezeichnung zu versehen. 

%	--------------------------------------------------------
% 	Quellenverzeichnis
%	--------------------------------------------------------
\section{Quellenverzeichnis}

Das Quellenverzeichnis enthält alle in der Arbeit zitierten (= in Fußnoten erwähnten) Quellen; im Text nicht zitierte Literatur ist nicht aufzuführen. Es ist nach Autorennamen zu sortieren. Versuchen Sie, möglichst aktuelle Literatur heranzuziehen. 

\textbf{Onlinequellen} (z.B. Internet, Online-Hilfe/-Dokumentation) führen Sie in einem \textbf{separaten} Abschnitt auf. Sie sind nur zu verwenden, wenn sich das Dokument in Papierform (z. B. Zeitschrift) nicht beschaffen lässt. 

\subsection{Grundsätzliche Angaben}
\begin{itemize}
	\item Name(n) des(r) Verfasser(s) (ohne akademische Grade)
	\item Vorname(n) ausgeschrieben 
	\item Titel des Werkes 
	\item Auflage (abgekürzt mit \glqq Aufl.\grqq), wenn es sich nicht um die erste Auflage handelt 
	\item Verlag (nicht bei Zeitschriften) 
	\item Erscheinungsort(e) (= Verlagsort(e)) (nicht bei Zeitschriften) 
	\item Erscheinungsjahr; zusätzlich bei Zeitschriften: Jahrgang und Heftnummer 
\end{itemize}

Beispiel:\\ 
\emph{Klaus Beuth: Digitaltechnik. 10. Aufl., Vogel Fachbuch, Würzburg 1998}


\subsection{Buch mit mehreren Verfassern}
Es werden alle Verfasser angegeben.


\subsection{Zeitschriften}
Folgende Angaben sind zusätzlich notwendig: \glqq In:\grqq hinter dem Titel des Aufsatzes, Name der Zeitschrift (übliche Abkürzung zulässig), Jahrgang, Kalenderjahr, Heft-Nr., Seite von \ldots bis \ldots 


\subsection{Kein Verfasser feststellbar}
Falls kein Verfasser feststellbar (z.B. Zeitungsartikel) ist, ist Anstelle eines Verfassernamens \glqq o.V.:\grqq zu verwenden. 

%	--------------------------------------------------------
% 	Beurteilungskriterien
%	--------------------------------------------------------
\section{Beurteilungskriterien}


\subsection{Grundlegende Beurteilungskriterien}
Der Beurteilung liegen vor allem folgende Kriterien zugrunde:
\begin{itemize}
	\item Umfang der gestellten Aufgabe und Grad der Innovation 
	\item Korrekte und einheitliche Verwendung der Fachausdrücke und Symbole 
	\item Eigene Lösungsvorschläge 
	\item Visualisierung von komplexen Zusammenhängen durch graphische/tabellarische Darstellungen 
	\item Sachliche Richtigkeit der Ausführung 
	\item Kreativität, Originalität 
	\item Hilfestellung ( musste viel Hilfe gegeben werden? ) 
	\item Präsentation 
	\item Einsatz und Teamarbeit 
	\item Formaler Aufbau der Dokumentation, Abstract, Illustration, Stil, Kontext 
	\item Gliederungssystematik, Gedankenführung (\glqq Roter Faden\grqq) 
	\item Rechtschreibung, Zeichensetzung, Zitierweise 
	\item Klarheit der Formulierung
	\item Eigenständigkeit
\end{itemize}


\subsection{DV-gestützte Arbeiten}
Zusätzlich für DV-gestützte Arbeiten:
\begin{itemize}
	\item Bedienerfreundlichkeit 
	\item Qualität der Dokumentation 
	\item Strukturierung des Programmcodes 
	\item Flexibilität, Erweiterungsmöglichkeiten, Portierbarkeit 
	\item Kommentare im Quellcode 
	\item Funktionalität (Demovorführung, Testplan inkl. nicht getestete Fälle, Abweichung von der Spezifikation) 
\end{itemize}


\subsection{Arbeiten mit Firmenbeteiligung:}
Zusätzlich für Arbeiten mit Firmenbeteiligung:
\begin{itemize}
	\item Beurteilung des Betreuers im Unternehmen 
	\item Einordnung des Themas in einen übergeordneten wissenschaftlichen (theoretischen) Zusammenhang 
\end{itemize}

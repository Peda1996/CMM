%!TEX root=../Vorlage_DA.tex
%	########################################################
% 				Buildprozess
%	########################################################


%	--------------------------------------------------------
% 	Allgmeine Hinweise
%	--------------------------------------------------------
\section{Buildprozess}

Um aus Quelltext ein ausf\"uhrbaren Anwendungsprogramm zu erstellen sind meist viele Schritte nötig.\footnote{\url{https://de.wikipedia.org/wiki/Erstellungsprozess}} Diese Aufgabe wird dabei in der Regel von einem sogenannten Build-Management-Tool \"ubernommen um Arbeitszeit zu sparen und die Entwicklung zu erleichtern.

\subsection{Apache Ant}

Apache Ant\footnote{\url{https://de.wikipedia.org/wiki/Apache_Ant}} ist ein in Java geschriebenes Build-Management-Tool welches auf Basis von XML-Dateien funktioniert.

Wir haben uns f\"ur Apache Ant entschieden da die Kompilierung von Java-Projekten darin einfach umsetzbar ist, und sich die Nutzung von XML-Dateien als Vorteil gegen\"uber die oft verwendeten Makefiles herausgestellt hat.

\subsubsection{Ausf\"uhren von Ant Prozessen}

Ein ausf\"uhrbarer Teil innerhalb der Build-Datei wird als target bezeichnet.

\begin{lstlisting}[language=XML]
<!-- clean project and then build it from scratch -->
<target name="all"
	depends="clean, build, test"
	description="build project from scrach"/>
\end{lstlisting}

\begin{lstlisting}[language=bash]
ant all
\end{lstlisting}
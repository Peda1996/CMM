\section{Interviews mit Lehrpersonen}
Anschließend an die Versuche haben wir Interviews mit den jeweiligen Lehrern geführt, um unsere Entwicklungsumgebung deren Bedürfnissen anzupassen. 

\subsection{Einleitung}
C Compact soll im Programmierunterricht nicht nur den Schülern helfen, sondern mit dem neuen Questsystem auch den Lehrer bei seinen Aufgaben unterstützen. Für die Planung und Implementierung dieser Komponente ist es wichtig, die verwendeten Lehrmethoden zu kennen und zu berücksichtigen. Wir haben beschlossen, die unterschiedlichen Unterrichtsmethoden an unserer Schule im Rahmen von Interviews mit FSST-Lehrern (Fachspezifische Softwaretechnik) zu erfassen und später zu vergleichen bzw. analysieren. 

Der FSST-Unterricht an unserer Schule wird gewissermaßen von Lehrer zu Lehrer weitergegeben:
Eine neue Lehrkraft orientiert sich anfangs meist an der Methode eines erfahrenen Kollegen. Dementsprechend liegen keine Aufzeichnungen über die verwendeten Lehrmethoden vor. 

Wir sind der Meinung, dass sich das verwendete System gut etabliert hat und wollen C Compact deshalb so an den Unterricht anpassen, dass es die Lehrkräfte in ihren bewährten Methoden unterstützt.

Die geführten Interviews wurden in zwei Blöcke aufgeteilt. Der erste Block behandelt die Strukturierung des Unterrichts des befragten Lehrers. Im zweiten Block wird mit dem Lehrer diskutieren, wie C Compact speziell in seine Lehrmethoden integriert werden kann und welche Features dazu nötig sind.

\subsection{Grundsätzlicher Ablauf eines FSST Unterrichts}
Die verwendeten Unterrichtsmethoden sind in ihren Grundzügen ähnlich: Es gibt immer eine Einführungsphase, in der die Theorie des aktuellen Themas erklärt wird. Nun werden erste Aufgaben gemeinsam mit der Lehrperson auf Papier oder Computer gelöst. Danach folgt die Übungsphase, bei der die SchülerInnen selbstständig Aufgaben lösen. In dieser Phase soll C Compact zum Einsatz kommen.

Um daher Vergleiche aufstellen zu können, wurden mehrere Parameter der jeweiligen Unterrichtsmethode notiert, wie etwa Dauer eines Themenblockes (in Wochen oder Unterrichtseinheiten) oder der Aufbau der Aufgabenstellungen. Somit können später Aufgaben zu einem Themenblock als Beispiel erstellt werden und als Empfehlungen zur Erstellung weitergegeben werden.

\subsubsection*{Ergebnisse}

Die Aufgaben im Unterricht werden in einem aufsteigenden Schwierigkeitsgrad gestellt. Falls eine Person fertig ist, gibt es noch Zusatzaufgaben zu lösen, welche besonders schwierig sind. Nach Abschluss der Übungen verlangen manche Lehrer eine Kopie der Programme, um diese Bewerten zu können. Sobald die Schüler mit diesen fertig sind, dürfen sie mit dem nächsten Thema beginnen.
Solche Themen dauern üblicherweise mehrere Wochen, wobei diese Themen in mehrere Themenblöcke unterteilt sind.

Um Schüler welche Probleme beim aktuellen Stoffgebiet haben zu unterstützen, werden diese von den Lehrern beziehungsweise anderen Schülern unterstützt. Oftmals reichen auch Lösungshinweise als Hilfestellungen. Schüler mit Lernproblemen benötigen grundsätzlich auch mehr Zeit zum Lösen von Aufgaben. Deshalb sind wesentliche Aufgaben immer am Anfang eines Themenblocks zu finden.

Die persönliche Betreuung durch den Lehrer sollte in C Compact auf jeden Fall berücksichtigt, wenn möglich sogar gefördert werden.

\subsection{Verwendung von C Compact im Unterricht}
Dieser Teil der Befragung hat zwei wesentliche Ziele: Erstens wollen wir wissen, was die Lehrer von einem Questsystem erwarten und wie sie es verwenden würden. Damit können wir sicherstellen, dass wir unser System so entwickeln, wie es tatsächlich gebraucht wird.

Zweitens wollen wir unseren weiteren Projektweg festlegen. Es gibt bereits einige Ideen für weitere Features, wie etwa Netzwerkfunktionen, Konsolengrafik, etc. Im Rahmen des Interviews wollten wir herausfinden, welche dieser Funktionen am nützlichsten wären.

Grundsätzlich wären alle von uns befragen Lehrpersonen dazu bereit C Compact im Unterricht zu verwenden. Sie sind davon überzeugt, dass die von uns kreierte Entwicklungsumgebung und das Quest System von Nutzen sein könnten. Jedoch sollte bei der Erstellung der Quests von den Lehrpersonen zusammengearbeitet werden.

Die befragten Personen waren davon überzeugt, dass es vorteile hätte, Netzwerkfunktionen zu implementieren, was eine elektronische Aufgabenabgabe und eine zugehörige Bewertung ermöglicht. Durch dieses System sollte auch ein Versenden von Aufgaben durch die Lehrperson ermöglicht werden. Auch wurde vorgeschlagen, dass die Aktivitäten der Schüler besser überblickt werden sollen. Somit könnte überprüft werden, wer gerade Arbeitet, welche Aufgabe geöffnet ist, ob die Entwicklungsumgebung gerade aktiv ist und wer gerade Probleme hat. Weiters wurde auch die Implementierung von Textgrafiken und einer alternativen Konsole zur Ein- und Ausgabe vorgeschlagen.

Die Lehrpersonen sind insgesamt zuversichtlich, dass das derzeitige Questsystem Schüler beim Lernen motivieren könnte.
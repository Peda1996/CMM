\section{Erstellung einer Quest}
Zuerst sollten die Dateien erstellt werden welche vom Programm benötigt werden. Diese sind: ref.cmm,input.cmm und description.html.

Hierbei ist zu beachten dass input.cmm Inputdaten für die ref.cmm und die zu testende Datei erstellt.
\begin{lstlisting}[language=C]
#include <stdio.h>
#include <stdlib.h>

void main(){
	srand(time());
	printf("%d ", rand());	
}

\end{lstlisting}
Im der hier gezeigten input.cmm wird eine Zufallszahl erzeugt, diese kann nun in der ref.cmm eingelesen und weiterverarbeitet werden. 
\begin{lstlisting}[language=C]
	//Initialisierung des Zufallsgenerators
	srand(atoi(scanf()));
\end{lstlisting}

Da die hier gezeigte Zeile nun auch im Programm des Benutzers vorhanden sein muss, kann man eine default.cmm erstellen. Diese beinhaltet einen vordefinierten Quelltext. Darin sollte auch die vorher gezeigte Zeile enthalten sein, damit es bei der Benutzung der Quest zu keiner Verwirrung kommt.

Der Aufbau einer guten Beschreibung einer Quest ist oftmals nicht einfach. Hierfür müssen mehrere Punkte beachtet werden. 

\begin{itemize}
\item Die Beschreibung sollte einfach und verständlich sein. 
\item Zuerst sollte ein theoretischer Teil beschrieben werden, danach die Aufgabenstellung.
\item Es sollte versucht werden, dass für die Textblöcke die richtigen Klassen des Stylesheets verwendet werden, sodass ein duchgängiges Design entstehen kann.
\item Überschiften mit dem html Tag "<h1"> oder "<h2"> kennzeichnen.
\end{itemize}
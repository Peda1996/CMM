\section{Profile}
Da der Fortschritt eines Benutzers irgendwo gespeichert werden muss und ein Benutzer vielleicht irgendwann mit den Quests von vorne beginnen möchte oder sich mehrere Personen einen Computer teilen, gibt es Benutzerprofile. Dort werden der Fortschritt der Aufgaben und die Erfolge eines Benutzers gespeichert.

\subsection{Konzept}
Ein Profil ist das dynamische Gegenstück zu dem \textbf{statisch} mit dem Programm verknüpften Questsystem. Während Quest-Packages im Ordner \textbf{packages}, welcher sich direkt beim Programm befindet, abgelegt werden müssen, kann das Profil überall gespeichert werden. Somit kann das Profil ohne Probleme auf einem USB-Stick mitgenommen werden und auf einem beliebigen Computer, wo die benötigten Quest-Packete installiert sind, wieder geöffnet werden.

Wenn ein Profil an einem Computer geöffnet wird, an dem im Profil verzeichnete Quests nicht vorhanden sind, werden diese einfach nicht angezeigt und in den Profildaten nicht verändert.

\subsection{Dateien eines Profils}
Das Herzstück des Profils befindet sich in der Datei \textbf{profile.cp}. Diese ist in XML codiert. Darin sind Benutzername und Bildpfad sowie Statusinformationen zu den Quests vermerkt. Sobald ein Profilbild gesetzt wird, wird dieses in den Profilordner kopiert und als \textbf{avatar."<png">} abgespeichert. Danach wird der neue Avatar in der \textbf{profile.cp} vermerkt.

\begin{lstlisting}[language=XML]
<profile>
	<name>TestProfil</name>
	<profileimage>avatar.jpeg</profileimage>
	<state id="finished">
		<quest>01 Simples Hello World</quest>
		<package>01 Einstieg</package>
		<date>16-03-2015:15:37:361</date>
		<filepath>/home/peda/Arbeitsfläche/einstieg1.cmm</filepath>
		<token>Icon_Craft.xml</token>
	</state>
	...
</profile>
\end{lstlisting}
Wenn im Profil ein Profilbild gespeichert wird, wird dieses zuerst in den Profilordner mit einem angepassten Bildtitel kopiert. Nun wird das alte Profilbild entfernt. Sobald dies geschehen ist, wird im Profil das neue Profilbild vermerkt. Somit können keine Fehler aufgrund der falschen Codierung des Namens auftreten.

Für Errungenschaften, welche beim Fertigstellen vom Quests erreicht werden, wird die Ordnerstruktur bis zum Tokens-Ordner im Profil angelegt. Somit können Auszeichnungen auch dann angezeigt werden, wenn die Quest, für die der Benutzer die Errungenschaft erhalten hat, nicht mehr verfügbar ist.

Zu den zugehörigen Quests werden auch die Pfade zu den Benutzerprogrammen gespeichert. Somit können angefangene Quests wieder fortgesetzt werden.

Falls C Compact geschlossen wird und gerade eine Quest in Bearbeitung ist, wird diese im Profil mit dem Status \textbf{open} gekennzeichnet. Aus diesem Grund werden beim erneuten Öffnen des Profils automatisch die zuletzt verwendete Quest und die zugehörige \textbf{.cmm} Datei geladen.

\subsection{Variablen in der profile.java}
\begin{lstlisting}[language=JAVA]
	private String name;				//Profilname				
	private String profileimage;		//profilbild
	private String current;				//derzeit geöffnetes File
	private String profilePath;			//Pfad zum Profil
	private Quest quest;				//aktuelle Quest

	private String packagesPath;		//"packages" Ordner
	private List<Quest> profileQuests;	//Liste aller Quests im Profil
\end{lstlisting}
Die hier gezeigten Variablen werden größtenteils von der \textbf{profile.cp} bestimmt. 

%!TEX root=../Vorlage_DA.tex
%	########################################################
% 				Realisierte Lösungen
%	########################################################

\subsection{Basisfunktionen}

Um die kommunikation zwischen Programm und Interpreter herzustellen besitzt C-Compact eine kleine Anzahl von Basisfunktionen, mit denen ein C-Compact Programm nach außen kommunizieren kann.

\subsubsection{read}

Diese Funktion ließt jeweils 1. Zeichen vom Eingabestrom

\begin{lstlisting}[language=CMM]
char c = read();
\end{lstlisting}

\subsubsection{print}

Diese Funktion schreibt jeweils 1. Zeichen auf den Ausgabestrom

\begin{lstlisting}[language=CMM]
print('a'); // test
\end{lstlisting}

\subsubsection{printf}

Diese Funktion implementiert eine einfache printf Funktion, wie sie auch in C vorhanden  ist. Da diese Funktion eine nicht festgelegte Anzahl von \"ubergabeparamter besitzt, was C-Compact eigentlich nicht unterst\"utzt, ist diese zu den Basisfunktionen hinzugef\"ugt worden.

\begin{lstlisting}[language=CMM]
printf("int: %d\n", 10);
\end{lstlisting}

\subsubsection{length}

Ermittelt die L\"ange eines Strings

\begin{lstlisting}[language=CMM]
string s = "Hello World!";
int l = length(s);
\end{lstlisting}

\subsubsection{time}

Gibt den aktuellen Timestamp des Computers zur\"uck. Wird unter anderem f\"ur den Zufallszahlengenerator ben\"otigt.

\begin{lstlisting}[language=CMM]
int t = time();
\end{lstlisting}

%\subsubsection{__is_def_type__}

%\subsubsection{__assert__}
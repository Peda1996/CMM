%!TEX root=../Vorlage_DA.tex
%	########################################################
% 				Allgemeiner Teil (Theorie)
%	########################################################


%	--------------------------------------------------------
% 	Sprachspezifikation
%	--------------------------------------------------------
\newpage
\section{Sprachspezifikation}

\subsection{Pr\"aprozessor}

Unser Pr\"aprozessor erm\"oglicht die Nutzung mehrer Dateien und verf\"ugt \"uber die funktion einzelne Codeteile zu deaktivieren, wie es auch in C der Fall ist. Pr\"aprozessorargumente beginnen dabei jeweils mit einem Hashtag, gefolgt von der jeweiligen Direktive und einem oder mehreren Argumenten.

\subsubsection{\#include Direktive}

Mithilfe eines \#include k\"onnen andere Dateien eingebunden werden. Es gibt dabei 2. Arten von \#include, welche sich durch den jeweiligen Suchpfad unterscheiden. Aus Technischer Sicht kopiert der Pr\"aprozessor die jeweilige Datei an die Stelle wo das \#include geschrieben ist.

\htlParagraph{Suche im Standard-Include-Pfad}

Der Standard-Include-Pfad stellt der Ordner clib dar, in welchem sich die Standardbibliotek befindet.

\begin{lstlisting}[language=C]
#include <stdio.h>
\end{lstlisting}

\htlParagraph{Suche im aktuellen Verzeichnis}

Wenn selbstgeschriebene Programmdateien eingebunden werden sollen wird der Pfad relativ zum aktuellen Verzeichnis angegeben. So ist es m\"oglich gro\"o\ss{}ere Projekte auf Basis mehrer Dateien zu entwickeln.

\begin{lstlisting}[language=C]
#include "foo.cmm"
\end{lstlisting}

\subsubsection{\#define Direktive}

Es ist m\"oglich den w\"ahrend des Pr\"aprozessorvorganges Variablen zu definieren, welche aber nur im Pr\"aprozessor ausgewertet werden k\"onnen. Es ist nicht m\"oglich mithilfe eines \#define Quelltext zu ver\"andern, wie es in C m\"oglich ist!

Es ist m\"oglich einem Define einen Wert zuzuweisen, welcher eine Ganze Zahl sein muss. Falls kein Wert angegeben ist wird 1 angenommen.

\begin{lstlisting}[language=C]
#define __DEFINE_WITHOUT_VALUE__
#define __DEFINE_WITH_VALUE__ 0
\end{lstlisting}

Falls ein \#define mit dem gleichen Namen bereits exisitiert, wird dieses \"uberschrieben.

\subsubsection{\#undef Direktive}

Mithilfe eines \#undef kann eine definierte Variable gel\"oscht werden. Die Variable steht somit nicht mehr zur verf\"ugung, bis die Variable erneut definiert wird.

\begin{lstlisting}[language=C]
#undef __DEFINE_WHICH_IS_NOW_DELETED__
\end{lstlisting}

\subsubsection{\#ifdef, \#ifndef, \#else und \#endif Direktive}

Es ist m\"oglich abzufragen ob es eine bestimmte Variablendefinition gibt, oder nicht gibt. Diese Abfrage wird mit den Pr\"aprozessorargumenten \#ifdef bzw. \#ifndef eingeleitet, und muss mit einem \#endif enden. Falls es notwendig ist auch f\"ur das gegenargument code auszuf\"uhren kann dies mithilfe eines \#else eingeleitet werden.

Eine Variable gilt als definiert wenn sie mithilfe eines \#define erzeugt wurde, und einen Wert ungleich 0 besitzt.

\begin{lstlisting}[language=C]
#ifdef __SOME_DEFINE__
	// ... Do something if __SOME_DEFINE__ is defined
#else
	// ... Do something if __SOME_DEFINE__ is not defined
#endif
\end{lstlisting}

\subsubsection{Beispiel}

Der Pr\"aprozessor wird besonders daf\"ur ben\"otigt, dass Biblioteken bei mehrfachen \#include keine Fehler verursachen. Dazu ist es notwendig dass die Bibliotek bei mehrfachen \#include die nachfolgenden Ignoriert. Dies stellt ein standardkonstrukt in der C Programmierung dar.

\begin{lstlisting}[language=C]
#ifndef __CLIB_EXAMPLE__

	#define __CLIB_EXAMPLE__
	
	// ... some includes if required
	#include <stdio.h>

	// ... here is the executed code of the file

#endif /* __CLIB_EXAMPLE__ */
\end{lstlisting}

Zuerst wird ermittelt, ob die Bibliotek bereits eingebunden wurde (falls dies der Fall ist, ist die jeweilige Variable definiert), und \#ifndef igoriert infolge den folgenden Code bis zum \#endif.

Falls der Code aber das erste mal eingebunden wurde, ist die Variable (in diesem Beispiel $\_\_CLIB\_EXAMPLE\_\_$) noch nicht definiert worden. Folglich wird der Code welcher sich in \#ifndef befindet ausgef\"uhrt, wo unter anderem die jeweilige Variable definiert wird, welche einzigartig f\"ur die jeweilige Bibliotek sein muss.

\subsection{Kommentare}

Bereiche welche als Kommentar\footnote{\url{https://de.wikipedia.org/wiki/Kommentar_(Programmierung)}} deklariert sind, werden vom Pr\"aprozessor und vom Compiler ignoriert. Es gibt dabei 2. Arten von Kommentare.

\htlParagraph{Zeilenkommentar}

Ein Zeilenkommentar beginnt mit einem //, und endet mit dem Ende der Zeile.

\begin{lstlisting}[language=C]
// this is a simple line comment
\end{lstlisting}

\htlParagraph{Blockkommentar}

Ein Blockkommentar beginnt mit einem /* und enden bei dem ersten auftreten eines */.

\begin{lstlisting}[language=C]
/* this is a
   blockcomment */
\end{lstlisting}

\subsection{Datentypen}

Der gew\"ahlte Datentyp\footnote{\url{https://de.wikipedia.org/wiki/Datentyp}} gibt an welche Art von Daten gespeichert werden k\"onnen. Es gibt primitive Datentypen, welche gro\"ss{}teils auch Arithmetische Datenoperationen unterst\"utzen, und Zusammengesetzte Datentypen welche aus einem oder mehreren primitiven Datentypen aufgebaut sind.

\subsubsection{Primitive Datentypen}

\htlParagraph{void}

void bezeichnet keinen eigentlichen Typen, und ist nur f\"ur die Definition von Funktionen erlaubt, welche nichts zur\"uckgeben.

\begin{lstlisting}[language=C]
void foo() {
}
\end{lstlisting}

\htlParagraph{bool}

bool unterst\"utzt die beiden Wahrheitswerte true und false. Wenn ein int als bool ausgewertet wird, stellt 0 false dar, und ungleich 0 ist true.

\begin{lstlisting}[language=CMM]
bool b;

bool foo() {
	return true;
}
\end{lstlisting}

\htlParagraph{char}

Ein char stellt ein einzelnes alphanumerisches Zeichen, ein Leerzeichen oder das Sonderzeichen \textbackslash{}r, \textbackslash{}n, \textbackslash{}t, \textbackslash{}0, \textbackslash{}' oder \textbackslash{}\textbackslash{} dar.

\begin{lstlisting}[language=CMM]
char ch;

char foo() {
	return 'c';
}
\end{lstlisting}

\htlParagraph{int}

Ein int stellt eine ganzzahlige Zahl dar, welche einen Wert zwischen $-2147483648$ und $2147483647$ haben muss.

\begin{lstlisting}[language=CMM]
int i;

int foo() {
	return 1234;
}
\end{lstlisting}

\htlParagraph{float}

\begin{lstlisting}[language=CMM]
float f;

float foo() {
	return 1.2;
}
\end{lstlisting}

\htlParagraph{string}

\begin{lstlisting}[language=CMM]
string s;

string foo() {
	return "Hello World";
}
\end{lstlisting}

\subsubsection{Konstanten}

Konstanten sind Variablen, welche nicht ver\"andert werden k\"onnen. Der Wert muss dabei bei der Deklaration angegeben werden, und muss dem Datentyp der Konstante entsprechen (Typumwandlungen sind nicht zul\"assig!).

\begin{lstlisting}[language=CMM]
const int i = 1234;
\end{lstlisting}

\subsubsection{Strukturen}

Strukturen sind zusammengesetzte Datentypen welche aus 1. oder mehreren Datentypen bestehen. Eine definierte Struktur stellt einen neuer Datentyp dar, von welchem Variablen definiert werden k\"onnen, welche man auch an Funktionen \"ubergeben kann.

Strukturen k\"onnen nicht auf sich selbst verweisen, da es ansonsten eine endlosen Rekursion darstellen w\"urde. Desweiteren k\"onnen keine Werte bei der Definition einer Struktur angegeben werden.

\begin{lstlisting}[language=CMM]
struct Point {
    int x, y;
    string name;
}

Point p;
\end{lstlisting}

\subsubsection{Arrays}

Arrays sind Felder von Datentypen, wobei ein einzelnes Feld mithilfe eines sogenannten Index spezifiziert werden kann.

Falls ein Array in einer Funktion definiert wird, kann diesem ein initialisierungswert zugewiesen werden, welcher das Array mit definierten Werten f\"ullt.

\begin{lstlisting}[language=CMM]
char cArr[10];
int arr[5][5];
\end{lstlisting}

\subsubsection{Typumwandlung}

Es ist m\"oglich Datentypen in einen anderen Umzuwandeln. Dies kann einerseits implizit geschehen, oder muss explizit angegeben werden. Es ist nicht m\"oglich jeden Datentyp in jeden anderen umzuwandeln. Strukturen und Arrays können bis auf besondere ausnahmef\"alle nicht umgewandelt werden.

\htlParagraph{Implizite Typumwandlung}

Bei der Impliziten Typumwandlung wird diese automatisch w\"ahrend des Compiliervorganges durchgef\"urt.

\begin{lstlisting}[language=CMM]
float f = 1 + 2.5; // 1 is implicit converted to float
\end{lstlisting}

\htlParagraph{Explizite Typumwandlung}

Die explizite Typumwandlung ist besonders dann notwendig, falls ein Datentyp in einen anderen umgewandelt werden muss, welcher weniger Daten speichern kann als sein ursprungstyp.

\begin{lstlisting}[language=CMM]
char ch = (char)48; // explicite conversation of int to char
\end{lstlisting}

\subsection{Funktionen}

\subsubsection{Vorimplementierte Funktionen}

\subsection{Ausdrücke}


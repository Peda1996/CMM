%!TEX root=../Vorlage_DA.tex
%	########################################################
% 					Standards für Ingenieurprojekte
%	########################################################


%	--------------------------------------------------------
% 	Überschrift, Inhaltsverzeichnis
%	--------------------------------------------------------
\chapter{Standards für Ingenieurprojekte}

(Aus Rundschreiben Nr. 60 der BMBWK aus 1999 unter der Geschäftszahl 17.600/101-II/2b/99) 

%	--------------------------------------------------------
% 	Definition eines Ingenieurprojektes
%	--------------------------------------------------------
\section{Definition eines Ingenieurprojektes}

Ein Ingenieurprojekt versteht sich als abschließender Leistungsnachweis des gesamten Ausbildungsweges an einer höheren technisch-gewerblichen Lehranstalt. Es soll dem Schüler in komplexer und praxisnaher Form Gelegenheit zur Umsetzung und Vertiefung der in seiner Ausbildungszeit erworbenen Kenntnisse und Fähigkeiten an Hand von Aufgabenstellungen auf gehobenem technischen Niveau geben. Wesentliche Merkmale sind dabei selbständige Arbeit und die Realisierung eigener Ideen.

Ein Ingenieurprojekt ist eine von einem zwei- bis sechsköpfigen Schülerteam durchzuführende, in sich geschlossene Arbeit. Zu jedem Team gehört dabei ein hauptverantwortlicher Projektbetreuer, der ein Lehrer mit entsprechender Fachexpertise sein muss. Die Aufgabenstellung soll industriespezifischen oder gewerblichen Charakter haben und die Durchführung möglichst in Kooperation mit einem außerschulischen Partner erfolgen. Die Dauer eines Ingenieurprojektes beträgt mindestens 6 Monate während des letzten Ausbildungsjahres ). Neben den fachlichen Aufgaben und Analysen sollen umweltrelevante Fragestellungen, Aspekte des Produktdesign sowie Kalkulation und Marketingplanung miteingeschlossen werden. Integrierter Bestandteil eines Ingenieurprojekts ist eine möglichst professionelle Dokumentation und eine gut vorbereitete Präsentation, die sich moderner Technologien zur Veranschaulichung bedienen soll.


%	--------------------------------------------------------
% 	Das pädagogische Konzept
%	--------------------------------------------------------
\section{Das pädagogische Konzept}

Das pädagogische Konzept orientiert sich an Prinzipien, die in zwei Gruppen zusammengefasst werden können:

\begin{itemize}
	\item Die inhaltlichen Grundsätze orientieren sich am im Einzelfalle höchstmöglich erreichbaren Maß an Praxisnähe. Ingenieurprojekte definieren sich dabei über fachlich komplexe Problemstellungen, Orientierung am Stand der Technik, gewissenhafte Strukturierung, detaillierte Planung, begleitendes Management, eine ausführliche Dokumentation, die Einbindung moderner Präsentationstechniken sowie der Beachtung der Grundsätze der Qualitätssicherung.
	\item Im Bereich der Persönlichkeitsbildung werden in Ergänzung und Vertiefung zu den allgemeinen Bildungszielen die Schulung der Teamfähigkeit, die individuelle Förderung spezieller Begabungen, die intensive Erfahrung von Selbständigkeit und Eigenverantwortlichkeit, ein individuelles Zeitmanagement, die Stärkung des Selbstbewusstseins und die Freiwilligkeit der Arbeitsleistung in den Mittelpunkt gestellt.
	
\end{itemize}	
%	--------------------------------------------------------
% 	Didaktische Konsequenzen
%	--------------------------------------------------------
\section{Didaktische Konsequenzen}

Das Erreichen dieses pädagogischen Konzeptes erfordert in weiten Feldern eine Neugewichtung der Unterrichtsprinzipien. So werden das Prinzip des fächerübergreifenden Unterrichts, \glqq Team-teaching\grqq (insbesondere auch durch Lehrer verschiedener Fächergruppen), eine Verschiebung vom lehrerzentrierten zum schülerzentrierten Unterricht, das Heranführen zu zielorientiertem und strukturiertem Arbeiten, die Entwicklung eines Zeit- und Kostenbewusstseins sowie eine Methodenvielfalt der Wissensaneignung in dieser Phase das Unterrichtsgeschehen dominieren.

%	--------------------------------------------------------
% 	Das Ziel: Eine neue Qualität in der Ausbildung
%	--------------------------------------------------------
\section{Das Ziel: Eine neue Qualität in der Ausbildung}

Die oben angeführten Konzepte ermöglichen bei konsequenter Ausrichtung in ihrer Gesamtheit eine neue Qualität in der Ausbildung. Die  Durchführung eines Ingenieurprojekts hat das Ziel, dem einzelnen Schüler 

\begin{itemize}
	\item Fachkompetenz
	\item Methodenkompetenz
	\item Sozialkompetenz und
	\item Selbstkompetenz
\end{itemize}

in integrativer Form und in einer Intensität und Qualität zu vermitteln, die deutlich über das bisher mit klassischen Unterrichtsformen mögliche Maß hinausgeht.

%	--------------------------------------------------------
% 	Entstehungs- und Entscheidungsphase
%	--------------------------------------------------------
\section{Entstehungs- und Entscheidungsphase}

Möglichst vielfältige schulexterne Kontakte sind bereits im Vorfeld anzustreben. Projekte mit außerschulischen Partnern sind das primäre Ziel, werden aber nicht immer realisierbar sein. Bei rein innerschulischen Projekten sind vorzüglich solche mit schulischer Wertschöpfung anzustreben.

Themenstellungen sollen möglichst gegenstandsübergreifend erfolgen, um beim Schüler ein Höchstmaß an Lösungskompetenz für die Berufspraxis zu erreichen. Die Projektthemen müssen einen Realitätsbezug zum Berufsfeld des Fachbereiches aufweisen. Es muss gewährleistet sein, dass relevante Kompetenzen aus dem angestrebten Berufsfeld vertieft und umgesetzt werden.

Die engere Themenwahl sollte sich dabei möglichst am realen Bedarf in Wirtschaft und Gesellschaft orientieren.

Projekte mit hohem Innovationsgehalt sind besonders zu fördern.

Jedes in die engere Wahl kommende Projekt muss im Interesse eines erfolgreichen Abschlusses ernsthaften Machbarkeitsüberlegungen unterzogen werden. Darüber hinaus sollte nach Maßgabe der Möglichkeiten eine Marktanalyse erfolgen.

Neben Machbarkeitsüberlegungen, die eine grundsätzliche Realisierbarkeit sicherstellen sollen, ist auch die Durchführbarkeit der einzelnen Projektvorschläge gewissenhaft und aufrichtig zu prüfen. Ziel dieser Prüfung ist, dass letztlich jedes begonnene Projekt für den Schüler auf Grund seiner Vorbildung bewältigbar und mit den zur Verfügung stehenden Ressourcen auch durchführbar ist.

%	--------------------------------------------------------
% 	Vorbereitungsphase
%	--------------------------------------------------------
\section{Vorbereitungsphase}

Am Beginn steht die Bildung des Projektteams. Ein solches besteht aus 2 bis 6 Schülern und aus einem (oder mehreren) projektbetreuenden Lehrer, der über die notwendige Fachexpertise verfügen muss. Die Zusammenstellung des Teams kann nach verschiedenen Gesichtspunkten wie etwa Schülerselbstbestimmung, lehrergesteuert, problemorientiert oder auch nach Zufallsaspekten erfolgen. Wenn mehrere Lehrer als Betreuer in ein Projekt eingebunden sind, ist ein hauptverantwortlicher Projektbetreuer zu nennen.

Die Rahmenbedingungen (rechtliche Fragen, Normen, einschlägige Vorschriften…) sind in das Thema einzuarbeiten und in die Projektdokumentation aufzunehmen.

Ebenso haben Recherchen zum Projektthema und dem fachlichem Umfeld durch das Projektteam in angemessenem Umfang zu erfolgen.

%	--------------------------------------------------------
% 	Durchführungsphase
%	--------------------------------------------------------
\section{Durchführungsphase}

Die Durchführung des Projektes hat in Teamarbeit zu erfolgen, arbeitsteilige Kooperation ist das zentrale Lernziel. Jedem Mitglied des Projektteams sind dabei persönliche Arbeitsanteile klar zuzuordnen, die auch eine individuelle Beurteilung im Rahmen der Teamarbeit erlauben.

Als erste Arbeit ist nachweislich ein ausführlicher Projektplan zu erstellen. Ausgehend von der Aufgabenstellung muss er eine klare Definition der Projektziele sowie ein genaues Pflichtenheft beinhalten. Der zeitliche Aufwand für den gesamten Projektablauf ist möglichst realistisch abschätzen und die \glqq Meilensteine\grqq und Termine sind in einem Terminplan festzulegen. Ebenso hat der Projektplan genaue Abschätzungen hinsichtlich der benötigten und zur Verfügung stehenden Ressourcen wie etwa Raumbedarf, Personal, Hard-und Software, budgetäre Bedeckung oder Arbeitsmaterialien zu enthalten.

Die Durchführung des Projektes hat unter Einbeziehung und Nutzung moderner Technologien zu erfolgen.

Ebenso sollte eine möglichst weit reichende Einbindung der englischen Sprache angestrebt werden (Heranziehen englischsprachiger Fachliteratur, Recherchen im Internet, ev. streckenweiser Einsatz von Englisch als Arbeitssprache, so etwa bei Zwischenpräsentationen).

Die genaue Führung eines Projekttagebuches ist unabdingbar.

%	--------------------------------------------------------
% 	Abschlussphase
%	--------------------------------------------------------
\section{Abschlussphase und Projektdokumentation}

Ein wesentliches Ziel des Projektes ist eine ordentliche und ausführliche Projektdokumentation, die das Projekt in allen Phasen und Ergebnissen beschreibt. Als Grundlage für diese Dokumentation ist das Projekttagebuch heranzuziehen.
 
Das zu Beginn erstellte Pflichtenheft ist zwingender Bestandteil der Dokumentation.
Ebenso muss die Projektdokumentation eine Zusammenfassung in englischer Sprache (Abstract) im Umfang etwa einer A4-Seite beinhalten.

Empfohlen wird eine gebundene Dokumentation in mehrfacher Ausführung (Schule-Schüler-ggf. außerschulischer Projektpartner). Die Schule sollte ein Exemplar zu Archivzwecken vorsehen.
 
Die Nutzung allgemein zugänglicher Dokumentations- und Kommunikationsmedien (Internet, Publikationen in Fachliteratur, \ldots) wird nach Maßgabe der Möglichkeiten empfohlen.

%	--------------------------------------------------------
% 	Projektpräsentation
%	--------------------------------------------------------
\section{Projektpräsentation}

Die abschließende Projektpräsentation hat in einem fest vorgegebenem Zeitrahmen unter Einsatz zeitgemäßer Medien und Präsentationstechnik zu erfolgen. Jedes Mitglied des präsentierenden Projektteams hat sich dabei auf gut vorbereitete  Präsentationsunterlagen zu stützen, wobei aber auf freie Rede besonders Bedacht zu nehmen ist. Ebenso ist darauf zu achten, dass jedem Gruppenmitglied möglichst gleicher Zeitanteil bei der Präsentation zukommt.

Jedem einzelnen Teammitglied steht es frei, auf eigenen Wunsch seinen Projektanteil in englischer Sprache zu präsentieren. 

An Stelle von Englisch kann in allen Bereichen auch eine andere Fremdsprache treten.

%	--------------------------------------------------------
% 	Qualitätssichernde Maßnahmen
%	--------------------------------------------------------
\section{Qualitätssichernde Maßnahmen}

Jedes Projekt ist auch aus der Sicht der Qualitätssicherung zu begleiten, wobei die Verantwortung für die Beachtung der Qualitätssicherungspunkte (insbesondere der hier festgelegten Standards) einer konkreten Stelle zuzukommen hat. Diese Stelle wird im Normalfall entweder eine nicht in das Projekt involvierte Person (Abteilungsvorstand, Fachkollege, Experte...), mehrere Einzelpersonen oder im Idealfall ein ordentlich installiertes Gremium (\glqq Arbeitsgruppe Ingenieurprojekte\grqq...) sein. Nur in begründeten Ausnahmefällen kann diese Aufgabe dem Projektbetreuer selbst zukommen.

Die grundlegenden qualitätssichernden Maßnahmen dürfen sich nicht in der Funktionprüfung eines hergestellten Produktes erschöpfen, sondern müssen auch andere Aspekte der Qualitätssicherung berücksichtigen, wie etwa Design, Umweltverträglichkeit,  Ergonomie, Kostenbewusstsein sowie den eigentlichen Projekt-Prozess. 

Produkte jeglicher Art haben sich am Stand der Technik zu orientieren. Geräte, Vorrichtungen und Anlagen müssen den geltenden Sicherheitsstandards entsprechen.

%	--------------------------------------------------------
% 	Controlling
%	--------------------------------------------------------
\section{Controlling}

Während des gesamten Projektablaufes hat ein laufendes Projekt-Audit zu erfolgen, das unmittelbares Reagieren auf unvorhergesehen auftretende Probleme jeglicher Art und vor allem auf Verzug gegenüber dem vorgesehenen Projektplan ermöglicht. Empfohlen werden in dieser Hinsicht Haltepunkte verbunden mit Terminkontrolle und regelmäßiger Rückmeldung, insbesondere bei größeren Projekten und Projekten mit außerschulischen Partnern. 